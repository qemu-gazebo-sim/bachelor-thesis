\documentclass[../../monografia.tex]{subfiles}
\graphicspath{ {images/}{../images/}{../../images/} }

\begin{document}
\section{Motivation}

\subsection{Micro-controllers and Embedded Systems}

The usage of micro-controllers to develop applications for embedded systems is handled by a wide range of companies. In this scope, one of the most famous producer for those hardware is STMicroelectronics \cite{STMicroelectronics_23}.

With a wide variety of devices and different families, the STM32 microcontrollers from STMicroelectronics, or ST, are used in many applications of embedded systems. Examples of these applications include robot control, Internet of Things devices, and automation.

STM32 microcontrollers generally work with the ARM architecture and include a series of functionalities provided by the Hardware Abstraction Layer (HAL), which optimizes software development for these devices. Combined with its widespread market usability, this makes it easier and more convenient to implement the desired functionalities and utilities for embedded systems.

Despite their high presence in this environment, ST microcontrollers lack a way to test all implemented software without using the physical device itself, which poses a barrier to development.

\subsection{Simulations and Robots}

In the realm of modern robotics, Gazebo applications are quite common. Gazebo \cite{gazebo_21} is a widely used open-source, multi-robot simulation environment that provides a 3D simulated world where robots, objects like sensors, and entire environments can be modeled and simulated with native plugins.

In the hardware simulation scenario, the open-source software QEMU \cite{QEMU_website_24} stands out. As a Kernel-based Virtual Machine (KVM), it serves as a generic machine emulator, leveraging Linux’s kernel to assist in the virtualization of different types of hardware. This usage allows for the emulation of microprocessor functionalities, enabling the simulation of embedded software execution within the emulated machine.

\subsection{Mobile Autonomous Robots}

With advances in technology and artificial intelligence, mobile autonomous robots are becoming increasingly common across various fields, from simple household tasks to space exploration. These robots can move independently without direct human intervention, enhancing efficiency and safety in many processes.

An example of a mobile autonomous robot application is the ‘robô hospitalar’, developed by the Escola Politécnica da Universidade de São Paulo. It assists in delivering medicines and necessary tools to hospital staff, improving logistics, efficiency, and people management. This field holds significant interest in robotics and the development of solutions for the benefit of humanity.

\end{document}