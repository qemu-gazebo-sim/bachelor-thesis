\documentclass[../monografia.tex]{subfiles}
\graphicspath{ {images/}{../images/}{../../images/} }

\begin{document}
\chapter{Requirements Specifications}
\label{chapter: Requirements Specifications}

This chapter presents the, functional and non-functional, requirements specifications of the project to be developed. 
\subsection{Functional}
\textbf{Micro-controller Emulation}: The project must be able to run the binary of an embedded system software the same way it would run in a real hardware. Therefore, the emulation of the functionalities and peripherals of the chosen micro-controller is essential.

\textbf{Environment and Robot simulation}: To reproduce how the physical system would respond in reality, it is necessary to have a whole simulated environment where the simulated robot will actuate.

\textbf{Connection Between the Emulation and the Simulation}: Since it is the simulation of a entire embedded system, it is necessary the connection between the hardware emulation and the environment simulation.

\textbf{Embedded System software binary}: To test and validate the project, it is necessary to have a functional software binary to be run by the hardware emulator.
\textbf{Physical embedded system}: To validate the project, it is essential to have a physical device. With that, it will be possible to see how close from a real system the simulation will get.

\subsection{Non-Functional}
\textbf{Simulation Performance}: It is important that the simulation has a good performance, reducing it's delay as maximum as possible.

\end{document}