\documentclass[../../monografia.tex]{subfiles}
\graphicspath{ {images/}{../images/}{../../images/} }


\begin{document}
    
    {\centering    
        \chapterNoPageBreakStart{Agradecimentos}
    }

Ao nosso professor orientador, Bruno Albertini, por seus conselhos e por sua paciência ao nos orientar durante todo o trabalho de conclusão de curso.

Aos nossos amigos Lucas Haug e Lucas Schneider, por terem sido nossos padrinhos no mundo do desenvolvimento de firmware e pelo companheirismo ao longo dos últimos anos.

À Universidade de São Paulo pelas oportunidades e a equipe de robótica Thunderatz, não apenas nos proporcionou uma base teórica e prática incrível durante os nossos primeiros cinco anos de graduação, mas também tornou nossa jornada acadêmica muito mais prazerosa e únicas.

\vspace{6mm}

\begin{itemize}
    \item Vanderson:
\end{itemize}

Primeiramente, à minha família, que me deram suporte, tanto emocional quanto financeiro durante toda a minha vida e graduação. Agradeço também por sempre acreditarem no meu potencial muito mais do que eu mesmo customo acreditar. 

À minha dupla de trabalho de conclusão de curso, Antonio, pelas noites mal dormidas estudando e trabalhando, tanto no TCC quanto em outras atividades acadêmicas. Agradeço também pela companhia nesses últimos 5 anos de graduação e por sempre me lembrar dos compromissos que eu nem lembrava que tinha.

\vspace{6mm}

\begin{itemize}
    \item Antonio:
\end{itemize}

Começo agradecendo aos meus pais João e Márcia, que sempre estiveram comigo, me incentivaram, acreditaram e me deram todo o suporte para que eu pudesse ir atrás dos meus sonhos. Ao meu irmão João, por todos os conselhos e guiamentos que sempre pude ter em qualquer que seja a situação, sendo peça fundamental para eu me tornar a pessoa que sou hoje. Ao meu irmão Pedro, o qual eu sei que posso contar com sua ajuda, não importa qual seja a situação. Agradeço aos meus avós, Jonas e Valdete, que sempre vibraram e continuam a vibrar por cada conquista que realizo.

Agradeço à minha dupla deste projeto, Vanderson, por compartilhar todo esse árduo processo de formação em Engenharia Elétrica. Obrigado por toda a parceria nesse período de graduação, todas as madrugadas realizando trabalhos, todas as competições de robótica partilhadas e por saber que sempre posso contar com você, seja para os momentos bons ou ruins.

\end{document}