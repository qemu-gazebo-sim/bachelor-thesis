\documentclass[../../monografia.tex]{subfiles}
\graphicspath{ {images/}{../images/}{../../images/} }

\begin{document}
\chapterNoPageBreak{Motivation}
\section{Micro-controllers and Embedded Systems}

The usage of micro-controllers to develop applications for embedded systems is handled by a wide range of companies. In this scope, one of the most famous producer for those hardware is the STMicroelectronics \cite{STMicroelectronics_23}.

With a big variety of devices and different families, the STM32 micro-controllers of STMicroelectronics , or just ST, are used in many applications of embedded devices. Examples of those applications are: robot controlling, Internet of Things devices and automation.

STM32 works in general with the ARM architecture and shows a series of functionalities to abstract the hardware with the Hardware Abstraction Layer ( HAL), which optimizes the software development with those devices. Joining this fact with the high usability in the market, it makes it interesting and simpler to implement the functionalities and utilities for those devices.

Despite their high presence in the environment of embedded systems, the micro-controllers of ST do not have a way to test all the software implemented without the physical device itself, which is a barrier to its development.

\section{Simulations and Robots}

In the universe of modern robotics, robotics applications made using Gazebo are very common. Gazebo is a widely used open-source, multi-robot simulation environment. It provides a 3D simulated world where robots, objects like sensors, and whole environments can be modeled and simulated with native plugins.

In the hardware simulation scenario, the open source software QEMU stands out. As a Kernel-based Virtual Machine, KVM, generic machine emulator, it gives the possibility to use Linux’s kernel to assist in the virtualization of different kinds of hardware.

With this usage, it is possible to emulate the functionalities of microprocessors, being able to simulate the execution of an embedded software in the emulated machine.

\section{Mobile Autonomous Robots}

With the advance of technology and artificial intelligence, mobile autonomous robots are becoming more common in a variety of fields, from simple house tasks to even space exploration. These robots are capable of moving independently without direct human intervention, which increases efficiency and safety in many processes.

An example of a application of a mobile autonomous robot is the “robô hospitalar”, developed by Escola Politécnica da Universidade de São Paulo, responsible to auxiliate the delivery of medicines and necessary tools to the hospital staff, helping in logistics, agility and people management. Therefore, it is a domain with great interest in robotics and the development of solutions for humanity.

\end{document}