\documentclass[../../monografia.tex]{subfiles}
\graphicspath{ {images/}{../images/}{../../images/} }

\begin{document}
\chapterNoPageBreak{Justification}

Nowadays, there are not so many options to simulate the functionality of an embedded system. Most of them are useful to test hardware features, one example is to use circuit emulators like Proteus Design Suite. Although it is possible to simulate some Microcontrollers with real source code, it is limited to the  elements and devices of the software itself, not being able to stipulate the data flow with a real device.

Other option is the use of Hardware-in-the-loop, which is a validation technique used to create a realistic testing environment. It is a good way to test complex systems applied to robotics, giving a safety and risk mitigation before the deploy of the application. On the other hand, although it is not necessary to have the complete physical setup, to use this technique it is still necessary to have hardware components to connect to the simulation environment. Therefore, it is impossible to test only the software that would be used itself.

Accordingly to this scope, in the current moment it is not possible to test a the software for the whole embedded system with all expected functionalities without a physical device. This brings a barrier to the development of an application as it brings the dependency of having a prototype and certifying that it is working correctly.

In this way, the implementation of this project will give another possibility for embedded systems software testing, being able to o code validation before the bench tests with a whole simulation for both the processor and the connected devices. As a result, it is expected to have: reducing on development costs, more safety and reliability on tests and reduce the number of evaluations on the real-world equipment.
 

\end{document}