% Configurações do pacote backref
\renewcommand{\backrefpagesname}{Citado na(s) página(s):~}
\renewcommand{\backref}{}
% Define os textos da citação
\renewcommand*{\backrefalt}[4]{
	\ifcase #1 %
		Nenhuma citação no texto.%
	\or
		Citado na página #2.%
	\else
		Citado #1 vezes nas páginas #2.%
	\fi}%

% Possibilita criação de Quadros e Lista de quadros.
% Ver https://github.com/abntex/abntex2/issues/176
\newcommand{\quadroname}{Quadro}
\newcommand{\listofquadrosname}{Lista de quadros}

\renewcommand{\cftquadroname}{\quadroname\space} 
\renewcommand*{\cftquadroaftersnum}{\hfill--\hfill}

\newcommand{\partNoPageBreak}[1]{%
  \bgroup
  \let\newpage\relax
  \part{#1}
  \egroup
}

\newcommand{\chapterNoPageBreak}[1]{%
  \bgroup
  \let\newpage\relax
  \chapter{#1}
  \egroup
}

\newcommand{\chapterNoPageBreakStart}[1]{%
  \bgroup
  \let\newpage\relax
  \chapter*{#1}
  \egroup
}

\newcommand{\partapendicesNoPageBreak}{%
  \bgroup
  \let\newpage\relax
  \partapendices
  \egroup
}