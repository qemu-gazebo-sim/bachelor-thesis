% ------------------------------------------------------------
% TEMPLATE
% ------------------------------------------------------------
\documentclass[
	% -- opções da classe memoir --
	12pt,				% tamanho da fonte
	openright,			% para impressão em recto e verso. Oposto a oneside
	a4paper,			% tamanho do papel.
        sumario=tradicional,
	% -- opções do pacote babel --
	french,				% idioma adicional para hifenização
	spanish,			% idioma adicional para hifenização
	brazil,				% o último idioma é o principal do documento
        english			% idioma adicional para hifenização
	]{../template/abntex2}

% ------------------------------------------------------------
% PACKAGE
% ------------------------------------------------------------ 
\usepackage{lmodern}			% Usa a fonte Latin Modern			
\usepackage[T1]{fontenc}		% Selecao de codigos de fonte.
\usepackage[utf8]{inputenc}		% Codificacao do documento (conversão automática dos acentos)
\usepackage{indentfirst}		% Indenta o primeiro parágrafo de cada seção.
\usepackage{color}				% Controle das cores
\usepackage{graphicx}			% Inclusão de gráficos
\usepackage{microtype} 			% para melhorias de justificação
		
% Pacotes adicionais, usados apenas no âmbito do Modelo Canônico do abnteX2
\usepackage{lipsum}				% para geração de dummy text

% Pacotes de citações
\usepackage[brazilian,hyperpageref]{backref}	 % Paginas com as citações na bibl
%\usepackage[alf]{abntex2cite}	% Citações padrão ABNT
\usepackage[num]{abntex2cite}
\usepackage{cite}

% Pacotes Locais
\usepackage{subfiles}
\usepackage{afterpage}

\usepackage{wrapfig} % Pacotes para figuras laterais
\usepackage{booktabs}
\usepackage{pdfpages}
\usepackage{listings}
\usepackage{minted}

\usepackage{color}

% ------------------------------------------------------------
% PACKAGES CONFIGURATION
% ------------------------------------------------------------ 
% ========== Images ==========
\graphicspath{ {images/}{../images/}{../../images/} }
% Informações de dados para CAPA e FOLHA DE ROSTO
\titulo{Embedded systems simulator\\ for robotics applications}
\autor{ Antonio Lago Araújo Seixas \\ Vanderson da Silva dos Santos}
\local{São Paulo, Brazil}
\data{2025}
\orientador{Prof. Dr. Bruno de Carvalho Albertini}
% \coorientador{Equipe \abnTeX}
\instituicao{%
  Universidade de São Paulo -- USP
  \par
  Escola Politénica
  \par
  Undergraduate Program}
\tipotrabalho{Final Paper (Undergraduate)}
% O preambulo deve conter o tipo do trabalho, o objetivo, 
% o nome da instituição e a área de concentração 
\preambulo{Work presented to Escola Politécnica da Universidade de São Paulo for undergraduate conclusion.}

% Configurações de aparência do PDF final

% alterando o aspecto da cor azul
\definecolor{blue}{RGB}{41,5,195}
\definecolor{black}{RGB}{0,0,0}

% informações do PDF
\makeatletter
\hypersetup{
     	%pagebackref=true,
		pdftitle={\@title}, 
		pdfauthor={\@author},
    	pdfsubject={\imprimirpreambulo},
	    pdfcreator={LaTeX with abnTeX2},
		pdfkeywords={abnt}{latex}{abntex}{abntex2}{trabalho acadêmico}, 
		colorlinks=true,       		% false: boxed links; true: colored links
    	linkcolor=black,          	% color of internal links
    	citecolor=blue,        		% color of links to bibliography
    	filecolor=magenta,      	% color of file links
		urlcolor=blue,
		bookmarksdepth=4
}
\makeatother

% Posiciona figuras e tabelas no topo da página quando adicionadas sozinhas
% em um página em branco. Ver https://github.com/abntex/abntex2/issues/170
\makeatletter
\setlength{\@fptop}{5pt} % Set distance from top of page to first float
\makeatother

% \newfloat[chapter]{quadro}{loq}{\quadroname}
\newlistof{listofquadros}{loq}{\listofquadrosname}
\newlistentry{quadro}{loq}{0}

% configurações para atender às regras da ABNT
\setfloatadjustment{quadro}{\centering}
\counterwithout{quadro}{chapter}

\setfloatlocations{quadro}{hbtp} % Ver https://github.com/abntex/abntex2/issues/176

% Espaçamentos entre linhas e parágrafos 
% O tamanho do parágrafo é dado por:
\setlength{\parindent}{1.3cm}

% Controle do espaçamento entre um parágrafo e outro:
\setlength{\parskip}{0.2cm}  % tente também \onelineskip



                           

\lstset{
    language=C++,                 % Set the language to C++
    basicstyle=\ttfamily,         % Use a monospaced font
    keywordstyle=\color{blue},    % Color keywords in blue
    commentstyle=\color{green},   % Color comments in green
    stringstyle=\color{red},      % Color strings in red
    numbers=left,                 % Add line numbers on the left
    numberstyle=\tiny,            % Use a tiny font for line numbers
    frame=single,                 % Add a simple frame around the code
    breaklines=true,              % Allow breaking long lines
    tabsize=4,                    % Set tab width to 4 spaces
    showstringspaces=false        % Do not show spaces in strings
}

% compila o indice
\makeindex
% ------------------------------------------------------------
% DOCUMENT COMMANDS
% ------------------------------------------------------------
% Configurações do pacote backref
\renewcommand{\backrefpagesname}{Citado na(s) página(s):~}
\renewcommand{\backref}{}
% Define os textos da citação
\renewcommand*{\backrefalt}[4]{
	\ifcase #1 %
		Nenhuma citação no texto.%
	\or
		Citado na página #2.%
	\else
		Citado #1 vezes nas páginas #2.%
	\fi}%

% Possibilita criação de Quadros e Lista de quadros.
% Ver https://github.com/abntex/abntex2/issues/176
\newcommand{\quadroname}{Quadro}
\newcommand{\listofquadrosname}{Lista de quadros}

\renewcommand{\cftquadroname}{\quadroname\space} 
\renewcommand*{\cftquadroaftersnum}{\hfill--\hfill}

\newcommand{\partNoPageBreak}[1]{%
  \bgroup
  \let\newpage\relax
  \part{#1}
  \egroup
}

\newcommand{\chapterNoPageBreak}[1]{%
  \bgroup
  \let\newpage\relax
  \chapter{#1}
  \egroup
}

\newcommand{\chapterNoPageBreakStart}[1]{%
  \bgroup
  \let\newpage\relax
  \chapter*{#1}
  \egroup
}

\newcommand{\partapendicesNoPageBreak}{%
  \bgroup
  \let\newpage\relax
  \partapendices
  \egroup
}

% ------------------------------------------------------------
% DOCUMENT BEGIN
% ------------------------------------------------------------
\begin{document}
% Seleciona o idioma do documento (conforme pacotes do babel)
\selectlanguage{english}
%\selectlanguage{brazil}

% Retira espaço extra obsoleto entre as frases.
\frenchspacing 

% ------------------------------------------------------------
% ELEMENTOS PRÉ-TEXTUAIS
% ------------------------------------------------------------
\imprimircapa

\imprimirfalsafolhaderosto

\imprimirfolhaderosto*

\subfile{preliminaries/pre_report}

% ------------------------------------------------------------
% LISTS
% ------------------------------------------------------------

% inserir lista de ilustrações
\clearpage
\pdfbookmark[0]{\listfigurename}{lof}
\listoffigures*
\clearpage

% inserir lista de quadros
% \clearpage
% \pdfbookmark[0]{\listofquadrosname}{loq}
% \listofquadros*

% inserir lista de tabelas
\clearpage
\pdfbookmark[0]{\listtablename}{lot}
\listoftables*
\clearpage

% \subfile{preliminaries/acronyms_symbols}

% ------------------------------------------------------------
% CONTENTS
% ------------------------------------------------------------
\pdfbookmark[0]{\contentsname}{toc}
\tableofcontents
\cleardoublepage

% ----------------------------------------------------------
% CHAPTERS ELEMENTS
% ----------------------------------------------------------
\textual
\subfile{chapters/introduction}

\subfile{chapters/conceptual_aspects}

\subfile{chapters/methodology}

\subfile{chapters/requirements_specification}

\subfile{chapters/project_development}

\subfile{chapters/final_considerations}

% ----------------------------------------------------------
% ELEMENTOS PÓS-TEXTUAIS
% ----------------------------------------------------------
\postextual

% ----------------------------------------------------------
% REFERENCES
% ----------------------------------------------------------
% --- TITULOS DO SUMÁRIOS
%\blinddocument

% --- ABNT (requer ABNTeX 2) ---
%	http://www.ctan.org/tex-archive/macros/latex/contrib/abntex2
%\bibliographystyle{abntex2-num}
\bibliography{bibliography}

% ----------------------------------------------------------
% GLOSSARY
% ----------------------------------------------------------

% Consulte o manual da classe abntex2 para orientações sobre o glossário.
%\glossary

% ----------------------------------------------------------
% APPENDICES 
% ----------------------------------------------------------
\subfile{finals/appendices}

% ----------------------------------------------------------
% ATTACHMENTS
% ----------------------------------------------------------
% \subfile{finals/attachments}

%-----------------------------------------------------------
% INDICE REMISSIVO
%-----------------------------------------------------------
% \phantompart
\printindex

% ------------------------------------------------------------
% DOCUMENT END
% ------------------------------------------------------------
\end{document}